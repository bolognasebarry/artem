\usepackage{amsmath,amssymb,amsfonts,amsthm,mathtools}
\usepackage[margin=15mm]{geometry}
\usepackage[dvipsnames]{xcolor}
\usepackage{graphicx}
\usepackage{tcolorbox}
\usepackage[final]{pdfpages}
\usepackage{gauss}
\usepackage{relsize}
\usepackage{etoolbox}
\usepackage[obeyspaces]{url}
\usepackage[explicit]{titlesec}
% \usepackage[en-GB,showzone=false]{datetime2}
% \setlength\parindent{0pt}
\usepackage{parskip}

% From Toms preamble
\usepackage[framemethod=TikZ]{mdframed}
\usepackage{float} \restylefloat{table}
% \definecolor{peach}{RGB}{255,218,185} % Custom colours

\allowdisplaybreaks %lets math go over the page

%-------------------------------------------- COMMANDS
\newcommand{\sidebyside}[2]{

      \noindent\begin{minipage}[t]{.5\linewidth}
            #1
      \end{minipage}%
      \noindent\begin{minipage}[t]{.5\linewidth}
            #2
      \end{minipage}
       
      }
% the above places two things side by side centred use like this \sidebyside{thing one}{thing two}
%the blank lines are necessary, otherwise you must place a blank line before and after to avoid errors, it can create large gaps though if something ends a line above

\newcommand{\sidebysidebyside}[3]{
      
      \noindent\begin{minipage}[t]{.3\linewidth}
            #1
      \end{minipage}%
      \noindent\begin{minipage}[t]{.3\linewidth}
            #2
      \end{minipage}%
      \noindent\begin{minipage}[t]{.3\linewidth}
            #3
      \end{minipage}
      
      }
% the above places three things side by side centred use like this \sidebysidebyside{thing one}{thing two}{thing three}
%the blank lines are necessary, otherwise you must place a blank line before and after to avoid errors, it can create large gaps though if something ends a line above

\newcommand{\diff}{\, \mathrm{d}} %dx but correct

\newcommand{\ddiff}[1]{\, \frac{\mathrm{d}}{\mathrm{d}  #1}} %d/dx but correct

\newcommand{\limtoinfinity}[1]{\lim_{#1 \to \infty}} %simple macro

\newcommand{\limtoneginfinity}[1]{\lim_{#1 \to -\infty}} %simple macro

\newcommand{\abs}[1]{\left| #1 \right|} %simple macro


%augmented matrix line % this shit is wack it makes a long vertical line cause gauss package cant do augmented matrices
\makeatletter
\patchcmd\g@matrix
 {\vbox\bgroup}
 {\vbox\bgroup\normalbaselines}% restore the standard base line skip
 {}{}
\makeatother

\newcommand{\mline}[1][0pt]{%
  \hspace{-\arraycolsep}%
  \ifdim#1>0pt
    \dimen0=\ht\strutbox \dimen2=\dimen0
    \advance\dimen0 #1\relax
    \ht\strutbox=\dimen0
  \fi
  \smash{\strut\vrule} % the `\vrule` is as high and deep as a strut
  % since assignments to \ht\strutbox are global, we restore the height
  \ifdim#1>0pt
    \ht\strutbox=\dimen2
  \fi
  \hspace{-\arraycolsep}%
}



%number less chapters
\newcommand{\mychapter}[2]{
    \setcounter{chapter}{#1}
    \setcounter{section}{0}
    \chapter*{#2}
    \addcontentsline{toc}{chapter}{#2}
}

\renewcommand{\thesection}{\hspace{5pt} \arabic{section}.} %whole numbered section


% custom headings and things
\newcommand{\lecture}[1]{
    \setcounter{section}{#1 - 1}
    \section{Lecture}
}

\titleformat{\section}{\normalfont\Large\bfseries}{}{0em}{#1\thesection}


% environments
% THEOREM % numberless
\newenvironment{theorem}[1][]{%
\vspace{3mm}%
\phantomsection%
\addcontentsline{toc}{subsection}{#1}%
\ifstrempty{#1}%
{\mdfsetup{%
frametitle={%
\tikz[baseline=(current bounding box.east),outer sep=0pt]
\node[anchor=east,rectangle,fill=blue!20, rounded corners=0.8mm]
{\strut Theorem};}}
}%
{\mdfsetup{%
frametitle={%
\tikz[baseline=(current bounding box.east),outer sep=0pt]
\node[anchor=east,rectangle,fill=blue!20, rounded corners=0.8mm]
{\strut Theorem:~#1};}}%
}%
\mdfsetup{innertopmargin=10pt,linecolor=blue!20,%
linewidth=2pt,topline=true,%
frametitleaboveskip=\dimexpr-\ht\strutbox\relax
}
\begin{mdframed}[]\relax%
}{\end{mdframed}}



% DEFINITION % numberless
\newenvironment{definition}[1][]{%
\vspace{3mm}%
\ifstrempty{#1}%
{\mdfsetup{%
frametitle={%
\tikz[baseline=(current bounding box.east),outer sep=0pt]
\node[anchor=east,rectangle,fill=green!20, rounded corners=0.8mm]
{\strut Definition};}}
}%
{\mdfsetup{%
frametitle={%
\tikz[baseline=(current bounding box.east),outer sep=0pt]
\node[anchor=east,rectangle,fill=green!20, rounded corners=0.8mm]
{\strut Definition:~#1};}}%
}%
\mdfsetup{innertopmargin=10pt,linecolor=green!20,%
linewidth=2pt,topline=true,%
frametitleaboveskip=\dimexpr-\ht\strutbox\relax
}
\begin{mdframed}[]\relax%
}{\end{mdframed}}



% LEMMMA
\newenvironment{lemma}[1][]{%
\vspace{3mm}%
\ifstrempty{#1}%
{\mdfsetup{%
frametitle={%
\tikz[baseline=(current bounding box.east),outer sep=0pt]
\node[anchor=east,rectangle,fill=red!20, rounded corners=0.8mm]
{\strut Lemma};}}
}%
{\mdfsetup{%
frametitle={%
\tikz[baseline=(current bounding box.east),outer sep=0pt]
\node[anchor=east,rectangle,fill=red!20, rounded corners=0.8mm]
{\strut Lemma:~#1};}}%
}%
\mdfsetup{innertopmargin=10pt,linecolor=red!20,%
linewidth=2pt,topline=true,%
frametitleaboveskip=\dimexpr-\ht\strutbox\relax
}
\begin{mdframed}[]\relax%
}{\end{mdframed}}


% EXAMPLE
\newenvironment{example}[1][]{%
\vspace{3mm}%
\ifstrempty{#1}%
{\mdfsetup{%
frametitle={%
\tikz[baseline=(current bounding box.east),outer sep=0pt]
\node[anchor=east,rectangle,fill=orange!20, rounded corners=0.8mm]
{\strut Example};}}
}%
{\mdfsetup{%
frametitle={%
\tikz[baseline=(current bounding box.east),outer sep=0pt]
\node[anchor=east,rectangle,fill=orange!20, rounded corners=0.8mm]
{\strut Example:~#1};}}%
}%
\mdfsetup{innertopmargin=10pt,linecolor=orange!20,%
linewidth=2pt,topline=true,%
frametitleaboveskip=\dimexpr-\ht\strutbox\relax
}
\begin{mdframed}[]\relax%
}{\end{mdframed}}


% PROPOSITION
\newenvironment{proposition}[1][]{%
\vspace{3mm}%
\ifstrempty{#1}%
{\mdfsetup{%
frametitle={%
\tikz[baseline=(current bounding box.east),outer sep=0pt]
\node[anchor=east,rectangle,fill=orange!20, rounded corners=0.8mm]
{\strut Proposition};}}
}%
{\mdfsetup{%
frametitle={%
\tikz[baseline=(current bounding box.east),outer sep=0pt]
\node[anchor=east,rectangle,fill=orange!20, rounded corners=0.8mm]
{\strut Proposition:~#1};}}%
}%
\mdfsetup{innertopmargin=10pt,linecolor=orange!20,%
linewidth=2pt,topline=true,%
frametitleaboveskip=\dimexpr-\ht\strutbox\relax
}
\begin{mdframed}[]\relax%
}{\end{mdframed}}

% CLAIM % numberless
\newenvironment{claim}[1][]{%
\vspace{3mm}%
\ifstrempty{#1}%
{\mdfsetup{%
frametitle={%
\tikz[baseline=(current bounding box.east),outer sep=0pt]
\node[anchor=east,rectangle,fill=Lavender!50, rounded corners=0.8mm]
{\strut Claim};}}
}%
{\mdfsetup{%
frametitle={%
\tikz[baseline=(current bounding box.east),outer sep=0pt]
\node[anchor=east,rectangle,fill=Lavender!50, rounded corners=0.8mm]
{\strut Claim:~#1};}}%
}%
\mdfsetup{innertopmargin=10pt,linecolor=Lavender!50,%
linewidth=2pt,topline=true,%
frametitleaboveskip=\dimexpr-\ht\strutbox\relax
}
\begin{mdframed}[]\relax%
}{\end{mdframed}}


% NOTE
\newenvironment{note}%
{\begin{quote}\textbf{Note}:}%
{\end{quote}}


% OBSERVATION
\newenvironment{observation}%
{\begin{quote}\textbf{Obs}:}%
{\end{quote}}

% TOOL
\newenvironment{tool}%
{\begin{quote}\textbf{Tool}:}%
{\end{quote}}

% FACT
\newenvironment{fact}%
{\begin{quote}\textbf{Fact}:}%
{\end{quote}}

% IDEA
\newenvironment{idea}%
{\begin{quote}\textbf{Idea}:}%
{\end{quote}}

% Corollary
\newenvironment{corollary}%
{\begin{quote}\textbf{Corollary}:}%
{\end{quote}}

% Question
\newenvironment{question}%
{\begin{quote}\textbf{Q}:}%
{\end{quote}}

% % Remark
% \newenvironment{remark}%
% {\begin{quote}\textbf{Remark}:}%
% {\end{quote}}

% REMARK % numberless
\newenvironment{remark}[1][]{%
\vspace{3mm}%
\ifstrempty{#1}%
{\mdfsetup{%
frametitle={%
\tikz[baseline=(current bounding box.east),outer sep=0pt]
\node[anchor=east,rectangle,fill=NavyBlue!20, rounded corners=0.8mm]
{\strut Remark};}}
}%
{\mdfsetup{%
frametitle={%
\tikz[baseline=(current bounding box.east),outer sep=0pt]
\node[anchor=east,rectangle,fill=NavyBlue!20, rounded corners=0.8mm]
{\strut Remark:~#1};}}%
}%
\mdfsetup{innertopmargin=10pt,linecolor=NavyBlue!20,%
linewidth=2pt,topline=true,%
frametitleaboveskip=\dimexpr-\ht\strutbox\relax
}
\begin{mdframed}[]\relax%
}{\end{mdframed}}

% Notation
\newenvironment{notation}%
{\begin{quote}\textbf{Notation}:}%
{\end{quote}}


%shits gotta come last ffs
\usepackage{hyperref}
% \hypersetup{pdfauthor={Jake Moss}, pdfkeywords={s46409665 - jake.moss@uqconnect.edu.au}}
